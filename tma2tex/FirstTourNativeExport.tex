%% AMS-LaTeX Created with the Wolfram Language : www.wolfram.com

\documentclass{article}
\usepackage{amsmath, amssymb, graphics, setspace}

\newcommand{\mathsym}[1]{{}}
\newcommand{\unicode}[1]{{}}

\newcounter{mathematicaitemnumbered}
\newcounter{mathematicasubitemnumbered}
\newcounter{mathematicasubsubitemnumbered}
\begin{document}

\title{Theorema 2.0: A First Tour}
\author{}
\date{}
\maketitle

This is a Theorema notebook. \pmb{ If this notebook is displayed within the Help-system, it is not possible to perform any actions in the notebook!}
In order to follow the examples in this tutorial {``}live{''} in your Theorema installation, we recommend to open the Theorema notebook {``}FirstTour.nb{''}
in the directory {``}TheoremaNotebooks{''} in the { }documentation directory of your Theorema installation using the {``}Open{''}-button in the Theorema
Commander.\\
 \pmb{ You can still browse through the notebook within the Help-system and it will be fully functional (after enabling dynamic objects).}\\
\hspace*{0.5ex} For Wolfram Workbench users: We recommend to move the entire directory {``}TheoremaNotebooks{''} including all subdirectories) from
the { }documentation directory of your Theorema installation into a location outside the Theorema Workbench project. Otherwise, the workbench will
notice a change within the project as soon as the auxiliary files in the associated directory are updated and start reloading the project over and
over.

We consider {``}proving{''}, {``}computing{''}, and {``}solving{''} as the three basic mathematical activities.

\section*{Proving}

We want to prove

\[\left(\underset{x}{\forall }(P[x]\lor Q[x])\right)\land \left(\underset{y}{\forall }(P[y]\Rightarrow Q[y])\right)\Leftrightarrow \left(\underset{x}{\forall
}Q[x]\right) .\]

To prove a formula like the above, we need to enter it in the context of a Theorema environment.



Proposition (First Test, 2014)

\(\left(\left(\underset{x}{\forall }(P[x]\lor Q[x])\right)\land \left(\underset{y}{\forall }(P[y]\Rightarrow Q[y])\right)\right)\Leftrightarrow \left(\underset{x}{\forall
}Q[x]\right)\)

$\blacksquare$



\(\fbox{\checkmark}\) Proof of \(\fbox{$\text{((P or Q $\land $ if P then Q) $\Leftrightarrow $ Q)}$}\) $\#$1: { } \(\fbox{$\underline{\text{Show
proof}}$}\)

\(\fbox{$\text{Unresolved} \text{Dynamic} \text{Content}$}\)

\noindent\(\text{Get}\text{::}\text{noopen}: "Cannot open \!\(\*RowBox[{\"\\\"/home/wwindste/Theorema.2/Theorema/Theorema/Documentation/English/TheoremaNotebooks/FirstTour/p169304498-1-po.m\\\"\"}]\)."\)

\noindent\(\text{Theorema}\text{::}\text{unexpectedArgs}: "Function \!\(\*RowBox[{\"Theorema`Provers`Common`Private`proofObjectToCell\"}]\) called
with unexpected arguments \!\(\*RowBox[{\"{\", \"Theorema`Provers`Common`Private`po$8077\", \"}\"}]\)."\)

\noindent\(\text{Throw}\text{::}\text{nocatch}: "Uncaught \!\(\*RowBox[{\"Throw\", \"[\", \"Theorema`Provers`Common`Private`proofObjectToCell\",
\"]\"}]\) returned to top level."\)

\noindent\(\text{Get}\text{::}\text{noopen}: "Cannot open \!\(\*RowBox[{\"\\\"/home/wwindste/Theorema.2/Theorema/Theorema/Documentation/English/TheoremaNotebooks/FirstTour/p169304498-1-po.m\\\"\"}]\)."\)

\noindent\(\text{Theorema}\text{::}\text{unexpectedArgs}: "Function \!\(\*RowBox[{\"Theorema`Provers`Common`Private`proofObjectToCell\"}]\) called
with unexpected arguments \!\(\*RowBox[{\"{\", \"Theorema`Provers`Common`Private`po$11051\", \"}\"}]\)."\)

\noindent\(\text{Throw}\text{::}\text{nocatch}: "Uncaught \!\(\*RowBox[{\"Throw\", \"[\", \"Theorema`Provers`Common`Private`proofObjectToCell\",
\"]\"}]\) returned to top level."\)

\noindent\(\text{Get}\text{::}\text{noopen}: "Cannot open \!\(\*RowBox[{\"\\\"/home/wwindste/Theorema.2/Theorema/Theorema/Documentation/English/TheoremaNotebooks/FirstTour/p169304498-1-po.m\\\"\"}]\)."\)

\noindent\(\text{Theorema}\text{::}\text{unexpectedArgs}: "Function \!\(\*RowBox[{\"Theorema`Provers`Common`Private`proofObjectToCell\"}]\) called
with unexpected arguments \!\(\*RowBox[{\"{\", \"Theorema`Provers`Common`Private`po$11065\", \"}\"}]\)."\)

\noindent\(\text{Throw}\text{::}\text{nocatch}: "Uncaught \!\(\*RowBox[{\"Throw\", \"[\", \"Theorema`Provers`Common`Private`proofObjectToCell\",
\"]\"}]\) returned to top level."\)

\section*{Computing}



Definition (Lexical Ordering)

\(\underset{| a| ==| b| }{\underset{a,b}{\forall }}\)

\(a<_{\text{lex}}b:\nthickspace \! \Longleftrightarrow \left(\underset{i=1,\ldots ,| a| }{\exists }\left(a_i<b_i\land \left(\underset{j=1,\ldots
,i-1}{\forall }\left(a_j==b_j\right)\right)\right)\right)\)

$\blacksquare$



\(\langle 1,1,1\rangle <_{\text{lex}}\langle 1,2,0\rangle\)

\subsection*{Domain Definitions}



Definition (Monomials)

\(\underset{K}{\forall }\)

\(\text{Mon}[K]\text{:=}\underset{M}{\Delta }\)

\(\underset{\text{m1},\text{m2}}{\forall }\)

\(\text{m1}\underset{M}{*}\text{m2}\text{:=}\left\langle \text{m1}_1\underset{K}{*}\text{m2}_1,\left\langle \left(\text{m1}_2\right)_i\underset{\mathbb{N}}{+}\left(\text{m2}_2\right)_i\underset{i=1,\ldots
,\left| \text{m1}_2\right| }{|}\right\rangle \right\rangle\)

\(\left(\text{m1}\underset{M}{<}\text{m2}\right):\nthickspace \! \Longleftrightarrow  \left(\text{m1}_2<_{\text{lex}}\text{m2}_2\right)\)

$\blacksquare$



\(\langle 3,\langle 1,2\rangle \rangle \underset{\text{Mon}[\mathbb{Q}]}{*}\langle 3/4,\langle 3,7\rangle \rangle\)

\(\langle 3,\langle 1,2\rangle \rangle \underset{\text{Mon}[\mathbb{Q}]}{<}\langle 3/4,\langle 3,7\rangle \rangle\)

\section*{Set Theory}



Definition (subset)

\(\underset{x,y}{\forall }\)

\(x\subseteq y\text{:=}\left(\underset{z}{\forall }(z\in x\Rightarrow z\in y)\right)\)

$\blacksquare$





Proposition (transitivity of $\subseteq $)

\(\underset{a,b,c}{\forall }((a\subseteq b\land b\subseteq c)\Rightarrow a\subseteq c)\)

$\blacksquare$



\(\mathsym{\WarningSign}\) Proof of \(\fbox{$\text{(4)}$}\) $\#$1: { } \(\fbox{$\underline{\text{Show proof}}$}\)

\(\fbox{$\text{Unresolved} \text{Dynamic} \text{Content}$}\)

\(\fbox{\checkmark}\) Proof of \(\fbox{$\text{(4)}$}\) $\#$2: { } \(\fbox{$\underline{\text{Show proof}}$}\)

\(\fbox{$\text{Unresolved} \text{Dynamic} \text{Content}$}\)

\end{document}
