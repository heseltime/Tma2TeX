\chapter{Abstract}

This work explores the Wolfram Language as a Software Engineering tool, with a particular focus on the Theorema mathematical software package, in combination with the \LaTeX typesetting system. It delves into the advanced functionalities and paradigms of Wolfram Language, including high-level programming, functional programming, and pattern matching, to showcase these capabilities beyond object oriented programming languages in particular, as applied to mathematical document transformation. 

Through Theorema,  package development using Wolfram Language is demonstrated from conception through execution to the point that the new package can be easily integrated with the existing Theorema system: the associated analysis touches on the workings of Theorema but the focus is on an implementational bridge between computational mathematics and document preparation, aiming to provide easy extensibility and delivering on further Software Engineering principles to make for a rounded Wolfram Language and Theorema package, as the final project output.

The thesis also addresses the challenges and methodologies associated with the \LaTeX typesetting of mathematical content, emphasizing the transformation of Wolfram Language/Theorema notebooks using a Wolfram-Language-native approach. This includes an examination of first-order predicate logic symbols, to ensure coverage at the output side, and the role of (mathematical) expressions in Wolfram Language, the input side, showcasing back-and-forth between typesetting and (symbolic) computational languages, and particularly, recursive parsing of entire notebook expressions as the basic working principle in this approach. 
