%% AMS-LaTeX Created with the Wolfram Language : www.wolfram.com

\documentclass{article}
\usepackage{amsmath, amssymb, graphics, setspace, xcolor}

% Define a command for light gray text for structure elements
\newcommand{\light}[1]{{\color{lightgray}#1}}

% Gray box for Tma envs
\usepackage{tcolorbox}

\newtcolorbox{tmaenvironment}{
  colback=gray!20, % Background color: gray with 20% intensity
  colframe=black, % Frame color: black
  boxrule=0, % Frame thickness
  arc=4pt, % Corner rounding
  boxsep=5pt, % Space between content and box edge
  left=5pt, % Left interior padding
  right=5pt % Right interior padding
}

\newtcolorbox{tmaenvironmentgd}{
  colback=gray!20, % Background color: gray with 20% intensity
  colframe=black, % Frame color: black
  boxrule=0, % Frame thickness
  arc=4pt, % Corner rounding
  boxsep=5pt, % Space between content and box edge
  left=5pt, % Left interior padding
  right=5pt % Right interior padding
}

% Define a custom light pastel purple color
\definecolor{lightpastelpurple}{RGB}{230, 220, 250}

\newtcolorbox{tmaenvironmentgd}{
  colback=lightpastelpurple!5, % Background color: pastel purple with 5% intensity 
    % currently not showing up maybe because of nesting
  colframe=black, % Frame color: black
  boxrule=0, % Frame thickness
  arc=4pt, % Corner rounding
  boxsep=5pt, % Space between content and box edge
  left=5pt, % Left interior padding
  right=5pt % Right interior padding
}

% gray square
\usepackage{tikz}

% for symbol encoding, e.g. "<", see also the following resource:
% https://tex.stackexchange.com/questions/2369/why-do-the-less-than-symbol-and-the-greater-than-symbol-appear-wrong-as
\usepackage{lmodern}

\newcommand{\graysquare}{\tikz\fill[gray] (0,0) rectangle (0.2cm,0.2cm);}


\newcommand{\mathsym}[1]{{}}
\newcommand{\unicode}[1]{{}}

\newcounter{mathematicapage}
\begin{document}

% \input{}

\title{Theorema 2.0: A First Tour}
\author{}
\date{}
\maketitle

\light{NB reached} \light{List of cells reached} \light{CellGroupData reached} \light{List of cells reached} Null\light{Cell reached} \begingroup \section*{} We consider “proving”, “computing”, and “solving” as the three basic mathematical activities.\endgroup 

\light{CellGroupData reached} \light{List of cells reached} \section{Proving}

\begingroup \section*{} We want to prove\endgroup 

\begin{center}(\underset{x}{\forall}(P[x] \lor Q[x])) \land (\underset{y}{\forall}(P[y] \Rightarrow Q[y])) \Leftrightarrow (\underset{x}{\forall}Q[x]) .\end{center}
\begingroup \section*{} To prove a formula like the above, we need to enter it in the context of a Theorema environment.\endgroup 

\begin{openenvironment}
\end{openenvironment}\begin{tmaenvironment}                     <>Null<>\end{tmaenvironment}
\subsection{Proposition (First Test, 2014)}
\light{Cell reached} \light{CellGroupData reached} \light{List of cells reached} \light{Cell reached} \light{Cell reached} \light{Cell reached} \light{Cell reached} \light{Cell reached} \light{Cell reached} \light{Cell reached} \light{Cell reached} \light{Cell reached} \light{Cell reached} \light{Cell reached} \light{CellGroupData reached} \light{List of cells reached} \section{Computing}

\begin{openenvironment}
\end{openenvironment}\light{CellGroupData reached} \light{List of cells reached} \light{Cell reached} \begin{tmaenvironmentgd}
\subsubsection{Global Declaration}
\underset{a = b}{\underset{a,b}{\forall}}\end{tmaenvironmentgd}
\begin{tmaenvironment}<>Null<>\end{tmaenvironment}
\subsection{[?]}
 \graysquare{}\light{Cell reached} \light{Cell reached} \light{CellGroupData reached} \light{List of cells reached} \light{Cell reached} \begin{openenvironment}
\end{openenvironment}\light{CellGroupData reached} \light{List of cells reached} \light{Cell reached} \begin{tmaenvironmentgd}
\subsubsection{Global Declaration}
\underset{K}{\forall}\end{tmaenvironmentgd}
\begin{tmaenvironmentgd}
\subsubsection{Global Declaration}
Mon[K]:=\underset{M}{\Delta}\end{tmaenvironmentgd}
\begin{tmaenvironmentgd}
\subsubsection{Global Declaration}
\underset{m1,m2}{\forall}\end{tmaenvironmentgd}
\begin{tmaenvironment}<>Null<>\end{tmaenvironment}
\subsection{[?]}
\begin{tmaenvironment}<>Null<>\end{tmaenvironment}
\subsection{[?]}
 \graysquare{}\light{Cell reached} \light{Cell reached} \light{Cell reached} \light{CellGroupData reached} \light{List of cells reached} \section{Set Theory}

\begin{openenvironment}
\end{openenvironment}\light{CellGroupData reached} \light{List of cells reached} \light{Cell reached} \begin{tmaenvironmentgd}
\subsubsection{Global Declaration}
\underset{x,y}{\forall}\end{tmaenvironmentgd}
\begin{tmaenvironment}<>Null<>\end{tmaenvironment}
\subsection{[?]}
 \graysquare{}\light{Cell reached} \begin{openenvironment}
\end{openenvironment}\begin{tmaenvironment}                              <>Null<>\end{tmaenvironment}
\subsection{Proposition (transitivity of \subseteq)}
\light{Cell reached} \light{CellGroupData reached} \light{List of cells reached} \light{Cell reached} \light{Cell reached} \light{CellGroupData reached} \light{List of cells reached} \light{Cell reached} \light{Cell reached} 

\end{document}