%% AMS-LaTeX Created with the Wolfram Language : www.wolfram.com
% from Wolfram template for LaTeX

\documentclass{article}

%% Packages
\usepackage{amsmath, amssymb, graphics, setspace, xcolor}

% for symbol encoding, e.g. "<", see also the following resource:
% https://tex.stackexchange.com/questions/2369/why-do-the-less-than-symbol-and-the-greater-than-symbol-appear-wrong-as
\usepackage{lmodern}

%% For development purposes
% Define a command for light gray text for structure elements
\newcommand{\light}[1]{{\color{lightgray}#1}}

%% Structural elements of LaTeX document
% Gray box for Tma envs
\usepackage{tcolorbox}

\newtcolorbox{tmaenvironment}{
  colback=gray!20, % Background color: gray with 20% intensity
  colframe=black, % Frame color: black
  boxrule=0, % Frame thickness
  arc=4pt, % Corner rounding
  boxsep=5pt, % Space between content and box edge
  left=5pt, % Left interior padding
  right=5pt % Right interior padding
}

\newtcolorbox{tmaenvironmentgd}{
  colback=gray!20, % Background color: gray with 20% intensity
  colframe=black, % Frame color: black
  boxrule=0, % Frame thickness
  arc=4pt, % Corner rounding
  boxsep=5pt, % Space between content and box edge
  left=5pt, % Left interior padding
  right=5pt % Right interior padding
}

% Define a custom light pastel purple color
\definecolor{lightpastelpurple}{RGB}{230, 220, 250}

\newtcolorbox{tmaenvironmentgd}{
  colback=lightpastelpurple!5, % Background color: pastel purple with 5% intensity 
    % currently not showing up maybe because of nesting
  colframe=black, % Frame color: black
  boxrule=0, % Frame thickness
  arc=4pt, % Corner rounding
  boxsep=5pt, % Space between content and box edge
  left=5pt, % Left interior padding
  right=5pt % Right interior padding
}

%% Strucutral elements of the symbol level
% gray square
\usepackage{tikz}
\newcommand{\graysquare}{\tikz\fill[gray] (0,0) rectangle (0.2cm,0.2cm);}

%% ---
%% Theorema Symbols are defined as custom LaTeX commands here! (BEGIN)
%%
%% In no particular order, except that it generally matches the parsing rules found in tma2tex.wl, Part 1.C.1,
%%  these commands define the syntax for theorema symbols in the LaTeX output. The naming convention is
%%  to use the symbol name in the Theorema code, but without dollar signs or context path. 
%%
%% So for example, the symbol
%%  TheoremaAnd$TM (with full context) becomes AndTM, and so on.
%% ---

%% ---
%% --- Approach 1 (Spring 2024)
%% ---

\newcommand{\IffTM}[2]{#1 \leftrightarrow #2}
\newcommand{\AndTM}[2]{#1 \wedge #2}

\newcommand{\ForallTM}[2]{\forall_{#1} #2}

\newcommand{\RNG}[1]{#1}
\newcommand{\SIMPRNG}[1]{\left( #1 \right)}

% VAR not directly translated
\newcommand{\VarTM}[1]{\mathit{#1}}

\newcommand{\OrTM}[2]{#1 \vee #2}

\newcommand{\Predicate}[2]{\textit{#1}\left(#2\right)} % TM2T-marker idea

\newcommand{\ImpliesTM}[2]{#1 \implies #2}

%% ---
%% --- Approach 2 (Summer 2024)
%% ---

\newcommand{\Variable}[1]{\textbf{#1}}

\newcommand{\Equivalent}[1]{\textbf{#1}}

%% ---
%% Theorema Symbols are defined as custom LaTeX commands here! (END)
%% ---

% From Wolfram template for LaTeX
\newcommand{\mathsym}[1]{{}}
\newcommand{\unicode}[1]{{}}

\newcounter{mathematicapage}
\begin{document}

% \input{}

\title{Theorema 2.0: A First Tour}
\author{}
\date{}
\maketitle

\light{NB reached} \light{List of cells reached} \light{CellGroupData reached} \light{List of cells reached} Null\light{Cell reached} \begingroup \section*{} We consider “proving”, “computing”, and “solving” as the three basic mathematical activities.\endgroup 

\light{CellGroupData reached} \light{List of cells reached} \section{Proving}

\begingroup \section*{} We want to prove\endgroup 

\begin{center}(\underset{x}{\forall}(P[x] \lor Q[x])) \land (\underset{y}{\forall}(P[y] \Rightarrow Q[y])) \Leftrightarrow (\underset{x}{\forall}Q[x]) .\end{center}
\begingroup \section*{} To prove a formula like the above, we need to enter it in the context of a Theorema environment.\endgroup 

\begin{openenvironment}
\end{openenvironment}\begin{tmaenvironment}
\subsection{Proposition (First Test, 2014)}
\ForallTM{x}{P(x}\lor Q(x))\land \ForallTM{y}{P(y}\Rightarrow Q(y))\unicode{29e6}\ForallTM{x}{Q(x})\end{tmaenvironment}
\light{Cell reached} \light{CellGroupData reached} \light{List of cells reached} \light{Cell reached} \light{Cell reached} \light{Cell reached} \light{Cell reached} \light{Cell reached} \light{Cell reached} \light{Cell reached} \light{Cell reached} \light{Cell reached} \light{Cell reached} \light{Cell reached} \light{CellGroupData reached} \light{List of cells reached} \section{Computing}

\begin{openenvironment}
\end{openenvironment}\light{CellGroupData reached} \light{List of cells reached} \light{Cell reached} \begin{tmaenvironmentgd}
\subsubsection{Global Declaration}
\underset{a = b}{\underset{a,b}{\forall}}\end{tmaenvironmentgd}
\begin{tmaenvironment}
\subsection{[?]}
\text{ForallTM}\left(a,b,\text{EquivalentDef}\left(\text{Annotated}(a,b),\exists _{\text{STEPRNG$\$$}(\text{arguments} \text{number} \text{of} \text{unexpected})}\left(a_i<b_i\land \text{ForallTM}\left(\text{STEPRNG$\$$}(\text{arguments} \text{number} \text{of} \text{unexpected}),a_j=b_j\right)\right)\right)\right)\end{tmaenvironment}
 \graysquare{}\light{Cell reached} \light{CellGroupData reached} \light{List of cells reached} \light{Cell reached} \light{Cell reached} \light{CellGroupData reached} \light{List of cells reached} \light{Cell reached} \light{Cell reached} \light{CellGroupData reached} \light{List of cells reached} \light{Cell reached} \begin{openenvironment}
\end{openenvironment}\light{CellGroupData reached} \light{List of cells reached} \light{Cell reached} \begin{tmaenvironmentgd}
\subsubsection{Global Declaration}
\underset{K}{\forall}\end{tmaenvironmentgd}
\begin{tmaenvironmentgd}
\subsubsection{Global Declaration}
Mon[K]:=\underset{M}{\Delta}\end{tmaenvironmentgd}
\begin{tmaenvironmentgd}
\subsubsection{Global Declaration}
\underset{m1,m2}{\forall}\end{tmaenvironmentgd}
\begin{tmaenvironment}<>Tma2tex`Private`formatTmaData[ Forall[unexpected number of arguments (Rng),  EqualDef[ DomainOperation[ m1,  m2],  Tuple[ DomainOperation[ Subscript[ m1, <>Tma2tex`Private`parseTmaData[1]<>],  Subscript[ m2, <>Tma2tex`Private`parseTmaData[1]<>]],  TupleOf[ STEPRNG$[unexpected number of arguments],  DomainOperation[ Subscript[ Subscript[ m1, <>Tma2tex`Private`parseTmaData[2]<>],  i],  Subscript[ Subscript[ m2, <>Tma2tex`Private`parseTmaData[2]<>],  i]]]]]]]<>\end{tmaenvironment}
\subsection{[?]}
\begin{tmaenvironment}<>Tma2tex`Private`formatTmaData[ Forall[unexpected number of arguments (Rng),  IffDef[ DomainOperation[ m1,  m2],  Annotated[ Subscript[ m1, <>Tma2tex`Private`parseTmaData[2]<>],  Subscript[ m2, <>Tma2tex`Private`parseTmaData[2]<>]]]]]<>\end{tmaenvironment}
\subsection{[?]}
 \graysquare{}\light{Cell reached} \light{CellGroupData reached} \light{List of cells reached} \light{Cell reached} \light{Cell reached} \light{CellGroupData reached} \light{List of cells reached} \light{Cell reached} \light{Cell reached} \light{CellGroupData reached} \light{List of cells reached} \light{Cell reached} \light{Cell reached} \light{CellGroupData reached} \light{List of cells reached} \light{Cell reached} \light{Cell reached} \light{CellGroupData reached} \light{List of cells reached} \section{Set Theory}

\begin{openenvironment}
\end{openenvironment}\light{CellGroupData reached} \light{List of cells reached} \light{Cell reached} \begin{tmaenvironmentgd}
\subsubsection{Global Declaration}
\underset{x,y}{\forall}\end{tmaenvironmentgd}
\begin{tmaenvironment}
\subsection{[?]}
\ForallTM{x}{y,\text{EqualDef}(x\subseteq y,\text{ForallTM}(z,z\in x\Rightarrow z\in y}))\end{tmaenvironment}
 \graysquare{}\light{Cell reached} \begin{openenvironment}
\end{openenvironment}\begin{tmaenvironment}
\subsection{Proposition (transitivity of \subseteq)}
\text{ForallTM}(\text{arguments} \text{number} \text{of} \text{Rng} \text{unexpected},a\subseteq b\land b\subseteq c\Rightarrow a\subseteq c)\end{tmaenvironment}
\light{Cell reached} \light{CellGroupData reached} \light{List of cells reached} \light{Cell reached} \light{Cell reached} \light{CellGroupData reached} \light{List of cells reached} \light{Cell reached} \light{Cell reached} 

\end{document}