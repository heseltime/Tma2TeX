%% AMS-LaTeX Created with the Wolfram Language : www.wolfram.com
% from Wolfram template for LaTeX

\documentclass{article}

%% Packages
\usepackage{amsmath, amssymb, graphics, setspace, xcolor}

% for symbol encoding, e.g. "<", see also the following resource:
% https://tex.stackexchange.com/questions/2369/why-do-the-less-than-symbol-and-the-greater-than-symbol-appear-wrong-as
\usepackage{lmodern}

%% For development purposes
% Define a command for light gray text for structure elements
\newcommand{\light}[1]{{\color{lightgray}#1}}

%% Structural elements of LaTeX document
% Gray box for Tma envs
\usepackage{tcolorbox}

%% Approach 1: Define a new tcolorbox environment for theorema environments
\newtcolorbox{tmaenvironment}{
  colback=gray!20, % Background color: gray with 20% intensity
  colframe=black, % Frame color: black
  boxrule=0, % Frame thickness
  arc=4pt, % Corner rounding
  boxsep=5pt, % Space between content and box edge
  left=5pt, % Left interior padding
  right=5pt % Right interior padding
}

%% Strucutral elements of the symbol level
% QED - gray square
\usepackage{tikz}
\newcommand{\graysquare}{\tikz\fill[gray] (0,0) rectangle (0.2cm,0.2cm);\hspace{0.2cm}}

% Centered diamond shape in a specified color
\newcommand{\colordiamond}[1]{%
    \tikz[baseline={([yshift=-0.3ex]current bounding box.center)}] 
    \fill[#1] (0,0) -- (0.08cm,0.08cm) -- (0,0.16cm) -- (-0.08cm,0.08cm) -- cycle;\xspace
}

% e.g.
% \colordiamond{red}
% \colordiamond{green}
% \colordiamond{blue}

% Define a command to create a right-aligned, smaller legend
\newcommand{\legend}{
  \begin{tcolorbox}[
    title=Tma2tex-parsing Info/Legend, 
    colback=white, 
    colframe=black, 
    boxrule=0.5mm, 
    arc=4pt, 
    width=6cm, % Adjust width to make it smaller
    float=right, % Moves the legend to the right
    halign=left, % Aligns the text within the box to the left
    left=2mm, % Interior left margin
    right=2mm, % Interior right margin
    boxsep=1mm % Reduces the separation between the text and the border
  ]
    \colordiamond{yellow} Yellow: Represents entry points to parsing. \\
    \colordiamond{orange} Orange: Helper Definitions were defined in the Theorema Notebook interface, but are directly included in the following formula. \\
    \colordiamond{red} Red: Matches unspecified cells or generic content. \\
    \colordiamond{blue} Blue: Represents lists of specific content. \\
    \colordiamond{purple} Purple: Used for lists of generic cells. \\
    \colordiamond{green} Green: Represents a CellGroupData Element with a List inside, a relevant content structure typically. \\
  \end{tcolorbox}
}


%% Approach 2: Dynamic environments
\usepackage{xparse} % To allow more flexible macro definitions

% Define a generic environment handler
\NewDocumentCommand{\EnvironmentWithFormat}{m m o}{%
    \IfValueTF{#3} % Check if there is an optional argument
    {%
        \begin{#1}[#3]%
        \textsc{#2}%
        \newline % Add a newline before the content
    }
    {%
        \begin{#1}%
        \textsc{#2}%
        \newline % Add a newline before the content
    }
}

% Define sample environments for demonstration
\newenvironment{proposition}[1][]{
  \noindent\textbf{Proposition #1:} 
}{\par}

\newenvironment{definition}[1][]{
  \noindent\textbf{Definition #1:} 
}{\par}

\newenvironment{lemma}[1][]{
  \noindent\textbf{Lemma #1:} 
}{\par}


%% ---
%% Theorema Symbols are defined as custom LaTeX commands here! (BEGIN)
%%
%% In no particular order, except that it generally matches the parsing rules found in tma2tex.wl, Part 1.C.1,
%%  these commands define the syntax for theorema symbols in the LaTeX output. The naming convention is
%%  to use the symbol name in the Theorema code, but without dollar signs or context path. 
%%
%% So for example, the symbol
%%  TheoremaAnd$TM (with full context) becomes AndTM, and so on.
%% ---

% Logical Operations
\newcommand{\IffTM}[2]{\left(#1 \iff #2\right)}
\newcommand{\AndTM}[2]{\left(#1 \land #2\right)}
\newcommand{\ImpliesTM}[2]{\left(#1 \rightarrow #2\right)}
\newcommand{\OrTM}[2]{\left(#1 \lor #2\right)}

% Quantifiers
\newcommand{\ForallTM}[2]{\forall #1 \, #2}
\newcommand{\ExistsTM}[2]{\exists #1 \, #2}

% Variables, Ranges, and Predicates
\newcommand{\RNGTM}[1]{#1}
\newcommand{\SIMPRNGTM}[1]{#1}
\newcommand{\STEPRNGTM}[1]{#1}
\newcommand{\VARTM}[1]{#1}

% Definitions
\newcommand{\IffDefTM}[2]{\left(#1 :\iff #2\right)}
\newcommand{\EqualDefTM}[2]{\left(#1 := #2\right)}

% Annotations and Subscripts
\newcommand{\AnnotatedTM}[3]{#1_{#2}\left(#3\right)}
\newcommand{\SubscriptTM}[2]{#1_{#2}}

% Operations and Relations
\newcommand{\LessTM}[2]{#1 < #2}
\newcommand{\EqualTM}[2]{#1 = #2}
\newcommand{\SubsetEqualTM}[2]{#1 \subseteq #2}

% Domain-specific Operations
\newcommand{\DomainOperationTM}[4]{#1_{#2}\left(#3, #4\right)}
\newcommand{\TupleTM}[2]{\left(#1, #2\right)}
\newcommand{\TupleOfTM}[2]{#1_{#2}}
\newcommand{\IntegerIntervalTM}[2]{[#1, #2]}

% Specific Mathematical Notations
\newcommand{\Mon}[1]{\text{Mon}\left[#1\right]}
\newcommand{\TimesTM}{\times}
\newcommand{\PlusTM}{+}

%% ---
%% Theorema Symbols are defined as custom LaTeX commands here! (END)
%% ---

% From Wolfram template for LaTeX
\newcommand{\mathsym}[1]{{}}
\newcommand{\unicode}[1]{{}}

\newcounter{mathematicapage}
\begin{document}

% \input{}

\title{Theorema 2.0: A First Tour}
\author{}
\date{}
\maketitle

% Insert legend at the Parsing Entry point, if DocumentProcessingLevel is so specified
\legend \n\n \colordiamond{yellow}\colordiamond{purple}\colordiamond{green}\colordiamond{purple}Null\colordiamond{red}\begingroup \section*{} We consider “proving”, “computing”, and “solving” as the three basic mathematical activities.\endgroup 

\colordiamond{green}\colordiamond{purple}\section{Proving}

\begingroup \section*{} We want to prove\endgroup 

\begin{center}(\underset{x}{\forall}(P[x] \lor Q[x])) \land (\underset{y}{\forall}(P[y] \Rightarrow Q[y])) \Leftrightarrow (\underset{x}{\forall}Q[x]) .\end{center}
\begingroup \section*{} To prove a formula like the above, we need to enter it in the context of a Theorema environment.\endgroup 

\colordiamond{red}\EnvironmentWithFormat{proposition}{First Test, 2014}
 \IffTM{ \AndTM{ \ForallTM{ \RNGTM{ \SIMPRNGTM{ \VARTM{x}}}}{ \OrTM{ P[ \VARTM{x}]}{ Q[ \VARTM{x}]}}}{ \ForallTM{ \RNGTM{ \SIMPRNGTM{ \VARTM{y}}}}{ \ImpliesTM{ P[ \VARTM{y}]}{ Q[ \VARTM{y}]}}}}{ \ForallTM{ \RNGTM{ \SIMPRNGTM{ \VARTM{x}}}}{ Q[ \VARTM{x}]}}
\end{proposition}
\colordiamond{red}\colordiamond{green}\colordiamond{purple}\colordiamond{red}\colordiamond{red}\colordiamond{red}\colordiamond{red}\colordiamond{red}\colordiamond{red}\colordiamond{red}\colordiamond{red}\colordiamond{red}\colordiamond{red}\colordiamond{red}\colordiamond{green}\colordiamond{purple}\section{Computing}

\colordiamond{red}\EnvironmentWithFormat{definition}{Lexical Ordering}
\colordiamond{orange}
 \ForallTM{ \RNGTM{ \SIMPRNGTM{ \VARTM{a}}}{ \SIMPRNGTM{ \VARTM{b}}}}{ \IffDefTM{ \AnnotatedTM{LessTM}{ \SubScriptTM{lex}}{ \VARTM{a}}{ \VARTM{b}}}{ \ExistsTM{ \RNGTM{ \STEPRNGTM}}{ \AndTM{ \LessTM{ \SubscriptTM{ \VARTM{a}}{ \VARTM{i}}}{ \SubscriptTM{ \VARTM{b}}{ \VARTM{i}}}}{ \ForallTM{ \RNGTM{ \STEPRNGTM}}{ \EqualTM{ \SubscriptTM{ \VARTM{a}}{ \VARTM{j}}}{ \SubscriptTM{ \VARTM{b}}{ \VARTM{j}}}}}}}}
\end{definition}
\colordiamond{red}\colordiamond{green}\colordiamond{purple}\colordiamond{red}\colordiamond{red}\colordiamond{green}\colordiamond{purple}\colordiamond{red}\colordiamond{red}\colordiamond{green}\colordiamond{purple}\colordiamond{red}\colordiamond{red}\EnvironmentWithFormat{definition}{Monomials}
\colordiamond{orange}
 \ForallTM{ \RNGTM{ \SIMPRNGTM{ \VARTM{K}}}{ \SIMPRNGTM{ \VARTM{m2}}}}{ \EqualDefTM{ \DomainOperationTM{ Mon[ \VARTM{K}]}{TimesTM}{ \VARTM{m1}}{ \VARTM{m2}}}{ \TupleTM{ \DomainOperationTM{ \VARTM{K}}{TimesTM}{ \SubscriptTM{ \VARTM{m1}}{1}}{ \SubscriptTM{ \VARTM{m2}}{1}}}{ \TupleOfTM{ \RNGTM{ \STEPRNGTM}}{ \DomainOperationTM{ \IntegerIntervalTM}{PlusTM}{ \SubscriptTM{ \SubscriptTM{ \VARTM{m1}}{2}}{ \VARTM{i}}}{ \SubscriptTM{ \SubscriptTM{ \VARTM{m2}}{2}}{ \VARTM{i}}}}}}}
\end{definition}
\colordiamond{red}\colordiamond{green}\colordiamond{purple}\colordiamond{red}\colordiamond{red}\colordiamond{green}\colordiamond{purple}\colordiamond{red}\colordiamond{red}\colordiamond{green}\colordiamond{purple}\colordiamond{red}\colordiamond{red}\colordiamond{green}\colordiamond{purple}\colordiamond{red}\colordiamond{red}\colordiamond{green}\colordiamond{purple}\section{Set Theory}

\colordiamond{red}\EnvironmentWithFormat{definition}{subset}
\colordiamond{orange}
 \ForallTM{ \RNGTM{ \SIMPRNGTM{ \VARTM{x}}}{ \SIMPRNGTM{ \VARTM{y}}}}{ \EqualDefTM{ \SubsetEqualTM{ \VARTM{x}}{ \VARTM{y}}}{ \ForallTM{ \RNGTM{ \SIMPRNGTM{ \VARTM{z}}}}{ \ImpliesTM{ \ElementTM{ \VARTM{z}}{ \VARTM{x}}}{ \ElementTM{ \VARTM{z}}{ \VARTM{y}}}}}}
\end{definition}
\colordiamond{red}\colordiamond{red}\EnvironmentWithFormat{proposition}{transitivity of ⊆}
 \ForallTM{ \RNGTM{ \SIMPRNGTM{ \VARTM{a}}}{ \SIMPRNGTM{ \VARTM{c}}}}{ \ImpliesTM{ \AndTM{ \SubsetEqualTM{ \VARTM{a}}{ \VARTM{b}}}{ \SubsetEqualTM{ \VARTM{b}}{ \VARTM{c}}}}{ \SubsetEqualTM{ \VARTM{a}}{ \VARTM{c}}}}
\end{proposition}
\colordiamond{red}\colordiamond{green}\colordiamond{purple}\colordiamond{red}\colordiamond{red}\colordiamond{green}\colordiamond{purple}\colordiamond{red}\colordiamond{red}

\end{document}