\subsection{Interchangeability of Theorema and General Wolfram Language Document Processing, Bigger Picture Considerations}

The transformation-program that this project resulted in is not specific to Theorema: it may be used for any WL-notebook (document) processing task, by applying the same principles: recursive descent, potential linkage to an underlying datastructure as in the Theorema-formula, high-level/built-in capabilites of WL for filehandling. (The underlying paradigms like functional programming are explored further in Chapter \ref{cha:Theory}.)

The principles surrounding pattern matching especially, allowing for efficient and expressive selection and processing of the relevant parts of a Mathematica notebook make systems where such a notebook is the start of a program-chain conceivable. The bigger picture, programmatically speaking, is this: what if the transformation output is not \LaTeX, but source code in say, Java, and the notebook triggers the execution after transformation, the way \LaTeX-to-PDF is triggered in the present work (see tma2tex.nb in the repository for how the call happens and tma2tex.wl for the implementation): WL provides such execution functionality at a conveniently high level, for Java \cite{noauthor_connect_nodate-1}, but also other sytems \cite{noauthor_connect_nodate}. The executed code can therefore be a dynamically constructed input, dependent only or mostly on a WL notebook as the leading system, allowing for highly-flexible broader systems making use of WL's high-level programming paradigm (see Section \ref{high-level} for an elaboration of this paradigm), and provides a footnote perspective of potential application where the output is not representation-oriented like \LaTeX, but an executable programming language.