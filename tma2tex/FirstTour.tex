%% AMS-LaTeX Created with the Wolfram Language : www.wolfram.com

\documentclass{article}
\usepackage{amsmath, amssymb, graphics, setspace, xcolor}

% Define a command for light gray text for structure elements
\newcommand{\light}[1]{{\color{lightgray}#1}}

% Gray box for Tma envs
\usepackage{tcolorbox}

\newtcolorbox{tmaenvironment}{
  colback=gray!20, % Background color: gray with 20% intensity
  colframe=black, % Frame color: black
  boxrule=0, % Frame thickness
  arc=4pt, % Corner rounding
  boxsep=5pt, % Space between content and box edge
  left=5pt, % Left interior padding
  right=5pt % Right interior padding
}

% gray square
\usepackage{tikz}

\newcommand{\graysquare}{\tikz\fill[gray] (0,0) rectangle (0.2cm,0.2cm);}


\newcommand{\mathsym}[1]{{}}
\newcommand{\unicode}[1]{{}}

\newcounter{mathematicapage}
\begin{document}

% \input{}

\title{Theorema 2.0: A First Tour}
\author{}
\date{}
\maketitle

\light{NB reached} \light{List of cells reached} \light{CellGroupData reached} \light{List of cells reached} Null\light{Cell reached} \begingroup \section*{} We consider “proving”, “computing”, and “solving” as the three basic mathematical activities.\endgroup 

\light{CellGroupData reached} \light{List of cells reached} \section{Proving}

\begingroup \section*{} We want to prove\endgroup 

\begin{center}(\forall x (P[x] \lor Q[x])) \land (\forall y (P[y] \Rightarrow Q[y])) \Leftrightarrow (\forall x Q[x]) .\end{center}
\begingroup \section*{} To prove a formula like the above, we need to enter it in the context of a Theorema environment.\endgroup 

\begin{openenvironment}
\end{openenvironment}\begin{tmaenvironment}
\subsection{Proposition (First Test, 2014)}
\left(\left(\forall x \left(P[x] \lor Q[x]\right)\right) \land \left(\forall y \left(P[y] \Rightarrow Q[y]\right)\right)\right) \Leftrightarrow \left(\forall x Q[x]\right) \graysquare{}\end{tmaenvironment}
\light{Cell reached} \light{CellGroupData reached} \light{List of cells reached} \light{Cell reached} \light{Cell reached} \light{Cell reached} \light{Cell reached} \light{Cell reached} \light{Cell reached} \light{Cell reached} \light{Cell reached} \light{Cell reached} \light{Cell reached} \light{Cell reached} \light{CellGroupData reached} \light{List of cells reached} \section{Computing}

\begin{openenvironment}
\end{openenvironment}\begin{tmaenvironment}
\subsection{Definition (Lexical Ordering)}
\light{Cell reached} aSubscriptBox[<, lex]bTagBox[RowBox[{:, , ⟺}], Identity, SyntaxForm -> ab]\left(\underset{i=1,…,a}{∃}\left(SubscriptBox[a, i]<SubscriptBox[b, i] \land \left(\forall j=1,…,i-1 \left(SubscriptBox[a, j]SubscriptBox[b, j]\right)\right)\right)\right) \graysquare{}\end{tmaenvironment}
\light{Cell reached} \light{Cell reached} \light{CellGroupData reached} \light{List of cells reached} \light{Cell reached} \begin{openenvironment}
\end{openenvironment}\begin{tmaenvironment}
\subsection{Definition (Monomials)}
\light{Cell reached} \light{Cell reached} \light{Cell reached} m1\underset{M}{*}m2:=〈SubscriptBox[m1, 1]\underset{K}{*}SubscriptBox[m2, 1],〈SubscriptBox[RowBox[{(, SubscriptBox[m1, 2], )}], i]\underset{}{+}SubscriptBox[RowBox[{(, SubscriptBox[m2, 2], )}], i]\underset{i=1,…,SubscriptBox[m1, 2]}{|}〉〉(m1\underset{M}{<}m2)TagBox[RowBox[{:, , ⟺}], Identity, SyntaxForm -> a⟺b] \left(SubscriptBox[m1, 2]SubscriptBox[<, lex]SubscriptBox[m2, 2]\right) \graysquare{}\end{tmaenvironment}
\light{Cell reached} \light{Cell reached} \light{Cell reached} \light{CellGroupData reached} \light{List of cells reached} \section{Set Theory}

\begin{openenvironment}
\end{openenvironment}\begin{tmaenvironment}
\subsection{Definition (subset)}
\light{Cell reached} x⊆y:=\left(\forall z \left(z∈x \Rightarrow z∈y\right)\right) \graysquare{}\end{tmaenvironment}
\light{Cell reached} \begin{openenvironment}
\end{openenvironment}\begin{tmaenvironment}
\subsection{Proposition (transitivity of ⊆)}
\forall a,b,c \left(\left(a⊆b \land b⊆c\right) \Rightarrow a⊆c\right) \graysquare{}\end{tmaenvironment}
\light{Cell reached} \light{CellGroupData reached} \light{List of cells reached} \light{Cell reached} \light{Cell reached} \light{CellGroupData reached} \light{List of cells reached} \light{Cell reached} \light{Cell reached} 

\end{document}