\chapter{Kurzfassung}

\begin{german}

Diese Arbeit untersucht die Wolfram Language als ein Software Engineering Werkzeug, mit einem besonderen Fokus auf das mathematische Softwarepaket Theorema in Kombination mit dem \LaTeX-Textsatzsystem. Sie vertieft sich in die fortgeschrittenen Funktionalitäten und Paradigmen der Wolfram Language, einschließlich der High Level Programmierung, der funktionalen Programmierung und des Pattern Matchings, um diese Fähigkeiten auch über die objektorientierte Programmierung hinaus zu demonstrieren, besonders im Hinblick auf die Transformation mathematischer Dokumente.

Mittels Theorema wird die Entwicklung von Packages in der Wolfram Language von der Konzeption über die Ausführung bis zu dem Punkt, an dem das neue Paket problemlos in das bestehende Theorema-System integriert werden kann, demonstriert. Die zugehörige Analyse befasst sich mit der Funktionsweise von Theorema, wobei der Schwerpunkt auf einer implementierungstechnischen Brücke zwischen computergestützter Mathematik und Dokumentenvorbereitung liegt. Das Ziel ist es, einfache Erweiterbarkeit zu ermöglichen und weitere Prinzipien des Software Engineerings zu realisieren, um ein umfassendes Wolfram und Theorema Language Package als Endprodukt des Projekts zu liefern.

Die Arbeit thematisiert auch die Herausforderungen und Methodologien, die mit dem \LaTeX-Textsatz von mathematischem Inhalt verbunden sind, und beleuchtet besonders die Transformation von Wolfram/Theorema-Notebooks nativ in der Wolfram Language. Dies beinhaltet eine Untersuchung von Symbolen der Prädikatenlogik erster Ordnung, um die Abdeckung auf der Ausgabeseite sicherzustellen, und die Rolle von (mathematischen) Ausdrücken (Expressions) in der Wolfram Language, der Eingabeseite, die die gegenseitige Kommunikation von zwischen Textsatz- und (symbolischer) Programmiersprache aufzeigt, insbesondere das rekursive Parsen von ganzen Notebook Expressions als grundlegendes Arbeitsprinzip des gewählten Ansatzes.

\end{german}
