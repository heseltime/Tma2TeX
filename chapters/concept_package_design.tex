\section{Package (Paclet) Design} \label{paclet-design}

\subsection{Paclet Design Generally}

The convention for designing packages in WL is very simple (see Program \ref{prog:PackageTemplate}). Native as well as third party packages [maybe examples] use this structure to effectively expose their API and configuration options. [... See also Programming Large Systems w/ WL]

For this Theorema package, the design idea is a modular extension that can be developed and maintained independently of Theorema itself, by defining it is an include [mechanism]. Alternatively, the code could be included directly in the Theorema package. [other includes? history?]

[Check implementation] The main functions are concealed by way of the `Private` context scoping mechanism, only exposing two at the top-level via the ::usage-decorator: convertToLatexDoc and convertToLatexAndPDFDocs. Their usage messages, callable (after loading with <<"Tma2tex`") via "?covertToLatexDoc" and "?convertToLatexAndPDFDocs" respectively, read: [...]

[Maybe a part about provisioning: think providing a cloud (connecting to berufspraktikum!) endpoint, compare to paclet repository approach (requiring download). Remark about generalizability of wl to tex transformation, independent of theorema language.]

\begin{program}
\caption{Template code for WL packages.}
\label{prog:PackageTemplate}
\begin{LaTeXCode}
BeginPackage["PackageName"]
$foo = "Foo"
Begin["`Private`"]
bar[] := bar1[] + bar2[]
End[]
EndPackage[]
\end{LaTeXCode}
\end{program}

\subsection{Paclet Design for this Project}

[TODO: Approach of Needs for texdump (permission to use/alt. approach?) ]
